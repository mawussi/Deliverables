\section{Concluding remarks}\label{sec:conclusion}
In this paper we have studied different algorithms for
the two-stage bidiagonalization with
a special focus on the first stage:
reduction from a full matrix to a band bidiagonal form.
We proved that even though the late update strategy
is simple to implement and follows the LAPACK-style,
it suffers from numerous synchronisation points that could be
released by the early update approach.

We also discuss potential improvements by relaxing
some OpenMP standard dependency expressing rules.
We demonstrated that while the improvement makes sense for the
late update approach,
the gain is not significant enough for the early update strategy
to sacrifice the robustness of the kernel.
Furthermore,
we have demonstrated the efficiency of our
implementation by showing that it is competitive
with the state-of-the-art kernels.

For the second stage:
reduction of band bidiagonal matrix to bidiagonal form,
we have identified two potential solutions which are
currently in development.

Since this work focused on square matrices,
future work will be devoted to designing specialized kernels
for very tall matrices
(number of rows more than 3x the number of columns)
which require different algorithms to keep the
computational resources working at full efficiency.
Also the current version of our prototype is based on OpenMP,
which is limited to shared memory systems.
In the future,
we will consider using distributed memory runtime
systems such as StarPU~\cite{AugThiNamWac11CCPE} and
PaRSEC~\cite{bosilca2013parsec}
to achieve an extreme-scale, heterogeneous implementation.
